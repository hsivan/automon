\section{Discussion} \label{sec:discussion}
AutoMon is an easy-to-use algorithmic building block for automatically approximating arbitrary real multivariate functions over distributed data streams.
Given a source code snippet of an arbitrary function of the global aggregate, AutoMon automatically provides communication-efficient distributed monitoring of the function approximation, without requiring any manual analysis by the user.
Our evaluation on synthetic and real-world datasets shows that AutoMon's error-communication tradeoff is comparable to previous hand-crafted algorithms, while using up to 50 times fewer messages on functions for which such efficient algorithms are not known.


Future work will concentrate on addressing AutoMon's limitations, and on improving its accuracy and performance.

First, AutoMon requires that $f$ be a function of the average vector $\bar{x}$, and does not capture functions such as $\sum_i \sum_j x^i x^j$ used in support vector machines~\cite{steinwart2008support}.
Though many functions can be rewritten in terms of $\bar{x}$~\cite{garofalakis2013sketch, lazerson:lightweight_monitoring, lazerson:one_for_all, gabel:entropy_approximation, papapetrou2014skylines, gabel:monitoring_least_squares}, %although
this is currently done manually.
We plan to explore automatic function rewriting, as well as support for more aggregations (e.g., max, sum of inner products).

Second, for functions not covered by the guarantees in \S\ref{sec:correctness_guarantees} the numerical optimization could yield inaccurate extreme eigenvalues and, therefore, violation of the error bounds.
Bounding the Hessian eigenvalues~\cite{2008_bound_hessian_eigenvalues,interval_matrix_branch_and_bound} can alleviate this issue. 
%
We also intend to study what factors impact AutoMon's performance.
The error-communication tradeoff is determined by both the function as well as the data and window size.
For example, when the extreme eigenvalues are exceptionally large/small, the derived safe zone can be very small leading to many safe zone violations.
Inferring this \emph{a priori} is hard for complex, hard-to-analyze functions -- if we could easily understand their behavior analytically, we would not need AutoMon in the first place.
However, we can use such observations to improve performance by switching on the fly to other monitoring approaches (e.g. Periodic).


Lastly, approximation error can be high if numerical optimization in the coordinator takes too long, which limits incoming data rate (\S\ref{sec:correctness_guarantees}, \S\ref{sec:scalability}).
To scale AutoMon to higher dimensions and data rates, we plan to explore Hessian spectrum approximations~\cite{lanczos_algo},
as well as pre-computing future constraints when the coordinator is idle.
